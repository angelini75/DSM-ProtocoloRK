\documentclass[10pt,a4paper]{article}
\usepackage[utf8]{inputenc}
\date{18 de agosto 2014}
\title{Un protocolo para la aplicación de modelos paramétricos en mapeo digital de suelos} 
\author{por Guillermo Federico OLMEDO y otros}
\usepackage{listings}
\usepackage{color}
\usepackage[hidelinks]{hyperref}
\definecolor{mygray}{gray}{0.95}
\lstset{language=R, numbers=left, frame=single, backgroundcolor=\color{mygray}, breaklines=true, numberstyle=\tiny\color{black}, basicstyle=\footnotesize\ttfamily,
  xleftmargin=\parindent, extendedchars=true, rulecolor=\color{white}, firstnumber=last}
%\lstset{language=R, belowcaptionskip=1\baselineskip,  frame=L,  xleftmargin=\parindent, basicstyle=\footnotesize\ttfamily}
\usepackage[spanish]{babel} %sudo apt-get install texlive-lang-spanish 
\usepackage[utf8]{inputenc}
\usepackage{natbib}
\bibpunct{(}{)}{;}{a}{,}{,}
\begin{document}
\maketitle
\section{Introducción}
El objetivo de este protocolo es sugerir una serie de pasos para aplicar metodologías de Mapeo Digital de Suelos\footnote{Digital Soil Mapping, en inglés.} a datos puntuales de suelos. En este primer protocolo se trabajará sobre la aplicación de un modelo paramétrico a variables continuas, conocido como \textit{regression-kriging}, o \textit{kriging with external trend} \citep{hengl_practical_2009}. 

Este es un modelo mixto, basado en la función de formación de suelos \citep{mcbratney_digital_2003} que relaciona el valor de una propiedad edáfica \(Sp\), con covariables ambientales que representan a los factores de formación de suelos: 
\begin{equation}
Sp = f(s,c,o,r,p,a,n)
\end{equation} 
Esta ecuación considera 7 factores: suelo \(s\); clima \(c\); organismos vivos \(o\); relieve \(r\); material parental \(p\); edad \(a\); y la posición \(n\). Este modelo esta compuesto por dos partes: una determinista y otra estocastica \citep{mcbratney_digital_2003}. La parte determinista consite en un modelo de regresión lineal múltiple entre el valor de la propiedad edáfica y las covariables ambientales. La parte estocástica contempla la porción de información edáfica no determinada por el modelo determinista (o sea, los residuos de la regresión) interpolados mediante un modelo geoestadístico de kriging ordinario.  Modelos no paramétricos y/o variables categóricas serán tratados en otros protocolos.

Este protocolo incluye el código necesario para ejecutar los pasos sugeridos en lenguaje R \citep{Rcit}. Además, con el objeto de facilitar la aplicación de este protocolo, todo el código utilizado se encuentra disponible en el archivo \textit{script.R} en el repositorio del protocolo: 
\url{https://github.com/midraed/DSM-ProtocoloRK}. En el mismo repositorio podrá encontrarse el conjunto de datos de prueba que se usará como ejemplo (\textit{santaines.csv}) y el archivo fuente en \LaTeX\space de este documento.

El código en este documento aparece en recuadros grises, y durante el texto se hará referencia a determinadas líneas a partir de la numeración de las mismas.
\begin{lstlisting}
setwd("directorio-de-trabajo")
library(gstat)
library(raster)
library(sp)
\end{lstlisting}
Los pasos sugeridos incluyen: preparación de los datos y  análisis descriptivo de los datos; ajuste y evaluación del modelo de regresión lineal múltiple; ajuste y evaluación del modelo geoestadístico de los residuos de regresión; combinación de los dos modelos para obtener el modelo de \textit{regression-kriging} y finalmente, validación del modelo obtenido.	

\section{El protocolo}
\subsection{Preparación de los datos}
\begin{lstlisting}
content...setwd("working_dir")
library(gstat)
library(raster)
library(sp)
library(soiltexture)
datos <- read.csv("Horiz_A.csv")
datos$Expandible <- as.factor(datos$Expandible)
datos.sp <- datos
coordinates(datos.sp) = ~X+Y
plot(datos.sp)
summary(datos.sp)
## Deben estar en una carpeta "covar" dentro de la carpeta de trabajo
## Cargamos las covariables ambientales en R
startdir <- getwd()
setwd(paste(getwd(), "/covar_ss", sep=""))
files <- list.files(pattern="sdat")
stack1 <- list()
for(i in 1:length(files)) {
stack1[[i]] <- raster(files[i])}
covariables <- do.call(stack, stack1) ### JO!
setwd(startdir)
covariables
plot(covariables)
covariables.sp <- as(covariables, "SpatialGridDataFrame") ## 45Mb!!!
## Creamos la grilla de interpolacion
halfres <- res(covariables)[1]/2
grilla <- expand.grid(x=seq(from=xmin(covariables)+halfres, to=xmax(covariables)-halfres, by=res(covariables)[1]), y=seq(from=ymin(covariables)+halfres, to=ymax(covariables)-halfres, by=res(covariables)[2]))
coordinates(grilla) <- ~ x+y
gridded(grilla) <- TRUE
# Extraemos los valores de las covariables
datos <- cbind(datos, extract(covariables, datos.sp))
\end{lstlisting}

\subsection{Análisis descriptivo}
\begin{lstlisting}
summary(datos)
plot(datos.sp,pch=1 ,cex = datos$Arcilla/max(datos$Arcilla)*4)
boxplot(datos$Arcilla)
summary(datos$Arcilla)
#Algunos graficos
boxplot(datos$Arcilla)
hist(datos$Arcilla)
#Diagrama de tallo-hoja
stem(datos$Arcilla)
#La Varianza
var(datos$Arcilla)
# Desvio standar
mean(datos$Arcilla)
sd(datos$Arcilla)
plot(datos.sp)
points(datos.sp[datos$Arcilla<30,])
plot(datos$Serie, datos$Arcilla, las=3, main="Contenido de Arcillas, horiz superficial", cex.axis=0.7)
\end{lstlisting}

\subsection{Modelo de regresión múltiple}
\begin{lstlisting}
Model.Arcilla.Super <- lm(Arcilla ~ Altitude_above_Channel_Network+Aspect+Channel_Network_Base_Level
+LS.Factor+Profile_Curvature+Slope_Height+Slope+SRTM30_estaca+Standardized_Height+Valley_Depth+Wetness_Index, data=datos)
summary(Model.Arcilla.Super)
Model.Arcilla.Super.step <- step(Model.Arcilla.Super, direction="both")
summary(Model.Arcilla.Super.step)
###
Model.Arcilla.Super2 <- lm(Arcilla ~ Altitude_above_Channel_Network+Channel_Network_Base_Level
+LS.Factor+Profile_Curvature+Slope+SRTM30_estaca+Wetness_Index, data=datos)
summary(Model.Arcilla.Super2)
Model.Arcilla.Super.step2 <- step(Model.Arcilla.Super2, direction="both")
summary(Model.Arcilla.Super.step2)
Model.Arcilla.Super.step2$residuals
Arcilla.Super.MLR <- predict(covariables, Model.Arcilla.Super.step2, progress="text")
plot(Arcilla.Super.MLR)
points(datos.sp)
\end{lstlisting}

\subsection{Modelo geoestadístico}
\begin{lstlisting}
#### Modelo de kriging ####
residuos.Arcilla.Sup <- Model.Arcilla.Super.step2$residuals
datos$Arcilla.res <- residuos.Arcilla.Sup
datos.sp <- datos
coordinates(datos.sp) = ~X+Y
plot(datos.sp,pch=1 ,cex = datos$Arcilla.res/max(datos$Arcilla.res)*4)
bubble(datos.sp, "Arcilla.res",col=c("blue", "red")) # Este graf. se ve mejor a pantalla completa
\end{lstlisting}

\subsection{Modelo de \textit{regression-kriging}}
\begin{lstlisting}
dependend_var.v <- variogram(as.formula(Model.Arcilla.Super.step2$call$formula), datos.sp)
dependend_var.ovgm <- fit.variogram(dependend_var.v, vgm(nugget=0, "Exp", range=sqrt(diff(datos.sp@bbox[1,])^2 + diff(datos.sp@bbox[2,])^2)/4, psill=var(Model.Arcilla.Super.step2$residuals)))
plot(dependend_var.v, dependend_var.ovgm, plot.nu=T, main="Residuos Arcilla Sup") ## 350*260
dependend_var.ovgm
Arcilla.Super.auto.rk <- krige(as.formula(Model.Arcilla.Super.step2$call$formula), datos.sp, covariables.sp, dependend_var.ovgm)
Arcilla.Super.rk.pred <- raster(Arcilla.Super.auto.rk["var1.pred"])
Arcilla.Super.rk.var <- raster(Arcilla.Super.auto.rk["var1.var"])
Arcilla.Super.rk.error <- qnorm(0.95)*sqrt(Arcilla.Super.rk.var)/sqrt(nrow(datos.sp))
plot(Arcilla.Super.rk.pred, main="Arcilla por RK")
points(datos.sp)
plot(Arcilla.Super.rk.pred-Arcilla.Super.rk.error, main="IC95 minimo") ## 
plot(Arcilla.Super.rk.pred+Arcilla.Super.rk.error, main="IC95 maximo")
writeRaster(Arcilla.Super.rk.pred, "Resultados/Arcilla.Super.RK.tiff", overwrite=TRUE)
\end{lstlisting}

\subsection{Validación}
\begin{lstlisting}
#### Leave One Out Cross Validation
Arcilla.Super.cross <- krige.cv(as.formula(Model.Arcilla.Super.step2$call$formula), datos.sp, dependend_var.ovgm , verbose=F)
Arcilla.Super.explained_variation <- 1-var(Arcilla.Super.cross$residual, na.rm=T)/var(datos.sp$Arcilla)
summary(Arcilla.Super.cross)
# mean error, ideally 0:
mean(Arcilla.Super.cross$residual)
# MSPE, ideally small
mean(Arcilla.Super.cross$residual^2)
# Mean square normalized error, ideally close to 1
mean(Arcilla.Super.cross$zscore^2)
# correlation observed and predicted, ideally 1
cor(Arcilla.Super.cross$observed, Arcilla.Super.cross$observed - Arcilla.Super.cross$residual)
# correlation predicted and residual, ideally 0
cor(Arcilla.Super.cross$observed - Arcilla.Super.cross$residual, Arcilla.Super.cross$residual)
# explained variation
Arcilla.Super.explained_variation
\end{lstlisting}

Y terminamos!
\bibliographystyle{plainnat}  
\bibliography{protocolo}
\end{document}